\textit{\textbf{1. } Dado un paralelogramo \textit{ABCD} en el plano afín, consideremos la familia de afinidades $f$ tal que $f$(\textit{A}) = \textit{B}, $f$(\textit{B}) = \textit{C} y $f$(\textit{C}) es un punto de la recta \textit{AD}.}

\textbf{(a) } \textit{Demostrad que hay una única afinidad de la familia sin puntos fijos y expresadla como la combinación de una homología seguida de una translación.}

\textbf{(b) } \textit{Demostrad que todas las otras afinidades de la familia tienen un punto fijo y encontrad el sitio geométrico de este punto.}
\\

\textbf{(a) } Consideremos la referencia $R = \{A; \Vec{AB}, \Vec{AC}\}$. Entonces tenemos que: 
\begin{gather*}
    A_R = (0, 0) \quad B_R = (1, 0) \quad C_R = (0, 1)
\end{gather*}

Sabiendo que las imágenes de $A, B, C$ son $B, C$ y un punto de la recta $AD$ respectivamente, tenemos:
\begin{gather*}
    f(A)_R = B_R = (1, 0)_R \\
    f(\Vec{AB})_R = f(B)_R - f(A)_R = C_R - B_R = (0, 1) - (1, 0) = (-1 ,1) \\
    f(\Vec{AC})_R = f(C)_R - f(A)_R = (-\lambda, \lambda) - (1, 0) = (-\lambda - 1, \lambda)
\end{gather*}
En la última de las igualdades anteriores se ha usado que $f(C) \in r := A + [\Vec{AD}]$ y que \textit{ABCD} es un paralelogramo. Por tanto,

\begin{gather*}
    \begin{cases}
    f(C) \in r := A + [\Vec{AD}]\\
    \Vec{AD} = \Vec{BC} \quad \text{(Regla del paralelogramo)}
    \end{cases}
    \Longrightarrow
    f(C) \in r := A + [\Vec{BC}] = (0,0)_R + [(-1,1)_R]
\end{gather*}

\vspace{0.5cm}
Resultando en:
\begin{gather*}
    f(\Vec{AC}_R) = f(C)_R - f(A)_R = (-\lambda, \lambda) - (1, 0) = (-\lambda - 1, \lambda)
\end{gather*}

Donde $(-\lambda, \lambda)$ es un punto genérico de la recta $r$.

\pagebreak

Una vez encontrada la imagen por $f$ de la base y del centro de la referencia $R$, podemos determinar la matriz genérica en función de $\lambda$ de esta familia de afinidades. Esta es:

\begin{gather}
    \widetilde{A} = 
    \begin{pmatrix}
    -1 & -\lambda -1\\
    1 & \lambda
    \end{pmatrix}
    \quad \quad
    b = 
    \begin{pmatrix}
    1\\ 0
    \end{pmatrix}
\end{gather}

Sabemos que una afinidad tiene puntos fijos $\Longleftrightarrow$  rango($\widetilde{A} - I$) = rango($\widetilde{A} - I | b$). Por tanto, calculemos el rango en función del parámetro $\lambda$.

\begin{gather*}
    (\widetilde{A} - I | b) =
    \begin{pmatrix}
    -2 & -\lambda - 1 & | & 1\\
    1 & \lambda - 1 & | & 0
    \end{pmatrix}
    \simeq
    \begin{pmatrix}
    0 & \lambda - 3 & | & 1\\
    1 & \lambda - 1 & | & 0
    \end{pmatrix}
    \simeq
    \begin{pmatrix}
    1 & \lambda - 1 & | & 0\\
    0 & \lambda - 3 & | & 1
    \end{pmatrix}
\end{gather*}

Como podemos observar, si $\lambda \neq 3 \Longrightarrow$ rango($\widetilde{A} - I$) = rango($\widetilde{A} - I | b$) = 2 y por consecuente las afinidades de la familia que cumplan esta condición contienen puntos fijos. Por otro lado, si $\lambda = 3 \Longrightarrow$ rango($\widetilde{A} - I$) = 1 $\neq$ 2 = rango($\widetilde{A} - I | b$). Así pues, existe una única afinidad $f$ de la familia que no contiene puntos fijos, la que cumple $\lambda$ = 3, cuya matriz en coordenadas ampliadas y en referencia $R$ es:

\begin{gather*}
    M(f, R) = 
    \begin{pmatrix}
    -1 & -4  & 1\\
    1 & 3  & 0\\
    0 & 0  & 1
    \end{pmatrix}
\end{gather*}

Una vez encontrada la matriz de la afinidad en cuestión, expresémosla ahora como la composición de una homología y una translación. Para eso, veamos si $\widetilde{A}$ diagonaliza. Calculemos el polinomio característico de $\Tilde{f}$.

\begin{gather*}
    P_{\Tilde{f}}(x) = 
    \begin{vmatrix}
    -1 - x & -4\\
    1 & 3 - x
    \end{vmatrix}
    = (-1-x)(3-x) + 4 = x^{2} -2x + 1 = (x - 1)^{2}
\end{gather*}

Esta afinidad consta de valor propio 1 con una multiplicidad algebraica 2. Calculemos su multiplicidad geométrica para comprobar si diagonaliza.

\begin{gather*}
    (\widetilde{A}- I)
    \begin{pmatrix}
    x\\ y
    \end{pmatrix}
    = 0 \Longrightarrow
    \begin{pmatrix}
    -2 & -4\\
    1 & 2
    \end{pmatrix}
    \begin{pmatrix}
    x\\ y
    \end{pmatrix}
    = 0 \Longrightarrow x + 2y = 0 \Longrightarrow (x, y) = (2, -1)
\end{gather*}

\pagebreak

Vemos que dim (Nuc($\widetilde{A} - I$)) = $m_g$ = 1 $\neq$ 2 = $m_a \Longrightarrow \widetilde{A}$ no diagonaliza. Pero, debido a que $P_{\Tilde{f}(x)}$ descompone completamente, por el segundo teorema de descomposición sabemos que existe una base de Jordan para el subespacio $Nuc(\widetilde{A} - I)^{2}$. Nos serviremos de esto ya que por el primer teorema de descomposición sabemos que $E = Nuc(\widetilde{A} - I)^{2}$, y por tanto, la base de Jordan anterior será también una base del espacio. Encontremos entonces una base de Jordan de $Nuc(\widetilde{A} - I)^{2}$.

Como nos encontramos en el plano afin (dim 2) y ya disponemos de uno de los dos vectores necesarios para formar la base, siendo este $v_1 = (2, -1) \in Nuc(\widetilde{A} - I)$, solamente tenemos que resolver el siguiente sistema lineal para encontrar el segundo vector de la base, $v_2 \in Nuc(\widetilde{A} - I)^{2}$.

\begin{gather*}
    (\widetilde{A} - I)v_2 = v_1 \Longrightarrow
    \begin{pmatrix}
    -2 & -4\\
    1 & 2
    \end{pmatrix}
    \begin{pmatrix}
    x\\ y
    \end{pmatrix}
    =
    \begin{pmatrix}
    2\\ -1
    \end{pmatrix}
    \Longrightarrow x + 2y = -1 \Longrightarrow v_2 = (-3, 1)
\end{gather*}

Observamos que $v_2 \notin Nuc(\widetilde{A} - I)$ porque no cumple la ecuación $x + 2y = 0$, pero en cambio es trivial ver que $v_2 \in Nuc(\widetilde{A} - I)^2$, ya que como $E = Nuc(\widetilde{A} - I)^2$ todos los vectores están en este subespacio. También vemos que $v_1,v_2$ son L.I y por tanto forman base. Además tenemos que $\widetilde{A}v_1 = v_1$ y que $\widetilde{A}v_2 = v_1 + v_2$. Sea entonces $R' = \{A; v_1, v_2\}$, la afinidad $f$ queda expresada de la siguiente manera:

\begin{gather*}
    f(x,y)_{R'} = 
    \begin{pmatrix}
    1 & 1\\
    0 & 1
    \end{pmatrix}
    \begin{pmatrix}
    x \\ y
    \end{pmatrix}
    +
    \begin{pmatrix}
    -1 \\ -1
    \end{pmatrix}
\end{gather*}

Siendo el vector (-1, -1) el vector (1, 0) expresado en la base $R'$. Ahora solamente nos queda hacer un último cambio de referencia para llegar a la matriz reducida de la afinidad. Sea $\Tilde{R} = \{A; u_1, u_2\}$, con $u_1 = (-1)v_1 = (-2, 1)$ y $u_2 = (-1)v_1 + (-1)v_2 = (1, 0)$, entonces la afinidad $f$ queda en la siguiente forma reducida en esta referencia:

\begin{gather}
    f(x, y)_{\Tilde{R}} = 
    \begin{pmatrix}
    1 & 1\\
    0 & 1
    \end{pmatrix}
    \begin{pmatrix}
    x\\ y
    \end{pmatrix}
    +
    \begin{pmatrix}
    0 \\ 1
    \end{pmatrix}
\end{gather}

Por teoría sabemos que esta es la expresión reducida de una homología especial compuesta con una translación. 

En conclusión, hemos visto que existe una única afinidad en la familia de afinidades definida en el enunciado que no tenga puntos fijos (la determinada cuando $\lambda$ = 3), y hemos llegado a expresarla como la composición de una homología especial con una translación, tal y como vemos en (2). Además, conocemos que la composición de una homología especial y una translación no tiene puntos fijos, que nos dice una vez más lo que ya conocíamos, que la afinidad $f$ no tiene ningún punto fijo.

\pagebreak

\textbf{(b) } Como hemos visto en el apartado anterior, si $\lambda \neq 3$, entonces la afinidad en cuestión contiene puntos fijos, por tanto, todas las afinidades de la familia menos la afinidad $f$ (la del apartado anterior) contienen algún punto fijo. Encontremos el lugar geométrico de estos puntos.

Conociendo la expresión de $\widetilde{A}$ y de $b$ en función de $\lambda$ que sabemos de (1), podemos resolver el siguiente sistema lineal compatible indeterminado para encontrar dónde se encuentran los puntos fijos de cada afinidad en función, una vez más, de $\lambda$ con $\lambda \neq 3$.

\begin{gather*}
    \widetilde{A}x + b = x \Longrightarrow 
    \begin{pmatrix}
    -1 & -\lambda - 1\\
    1 & \lambda
    \end{pmatrix}
    \begin{pmatrix}
    x\\ y
    \end{pmatrix}
    +
    \begin{pmatrix}
    1\\ 0
    \end{pmatrix}
    = 
    \begin{pmatrix}
    x\\ y
    \end{pmatrix}
    \\
    \\
    \Longrightarrow
    \begin{cases}
    -x -\lambda y - y + 1 = x\\
    x + \lambda y = y
    \end{cases}
    \Longrightarrow
    \begin{cases}
    x = \frac{\lambda - 1}{\lambda - 3}\\
    y = \frac{1}{3 - \lambda}
    \end{cases}
\end{gather*}

De modo que el punto fijo genérico de estas afinidades es $p = (x,y) = (\frac{\lambda - 1}{\lambda - 3}, \frac{1}{3 - \lambda})$. Observamos que, fijada una $\lambda$, el sistema anterior pasa a ser compatible determinado, y por consecuente el punto queda unívocamente determinado, por lo que cada una de estas afinidades tiene un único punto fijo. Observamos también lo siguiente:

\begin{gather*}
    \lambda - 1 = 2 + \lambda - 3 \Longrightarrow \quad \frac{\lambda - 1}{\lambda - 3}= \frac{2+\lambda-3}{\lambda-3} = 1 + \frac{2}{\lambda - 3}
\end{gather*}

Vemos que $\frac{1}{\lambda - 3}$ con $\lambda \neq$ 3 toma todos los valores reales menos el 0. Siendo $\alpha = \frac{1}{\lambda - 3}$, el conjunto de puntos fijos es $(1 + 2\alpha, -\alpha)$, que genera la recta $s := (1,0)+ [(2,-1)]$. Por tanto, hemos encontrado el lugar geométrico de los puntos fijos de  esta familia de afinidades. Este es $s\setminus\{(1,0)\}$ ya que, como $\forall \lambda \in \mathbb{R} \ \ \alpha = \frac{1}{\lambda - 3} \neq 0$, no podemos encontrar ninguna afinidad asociada a un $\lambda$ para el que (1,0), que correspnde al punto $B$, sea un punto fijo.

En conclusión, hemos visto que todas las afinidades de la familia menos la del apartado \textbf{(a)} tienen un único punto fijo $p = (\frac{\lambda - 1}{\lambda - 3}, \frac{1}{3 - \lambda})$, que queda unívocamente determinado fijado una $\lambda$. El conjunto de estos puntos genera $s$, que es la recta que contiene los puntos fijos de estas afinidades, exceptuando el (1,0) que es el único punto no fijo de $s$. Por tanto el lugar geométrico que se nos pide encontrar es $s\setminus\{(1,0)\}$.