\textbf{1. }

(\textbf{i}) \textit{Sea $F$ un giro de $\mathbb{E}^{2}_{\mathbb{R}}$ al rededor de un punto $p$. Si $F(q_1) = q_2$, demostrad que $p$ pertenece a la recta perpendicular a $u = q_2 - q_1$ y que contiene el punto $(q_1 + q_2)/2$. Como aplicación, encontrad el giro de $\mathbb{E}^{2}_{\mathbb{R}}$ que verifica $F(1,1) = (-1,3)$ y que $F(2,0) = (0,4)$ (ecuaciones, centro y ángulo de giro).}

(\textbf{ii}) \textit{Sea $F$ una rotación al rededor de una recta $r$ de $\mathbb{E}^{3}_{\mathbb{R}}$. Si $F(p) = q$, demostrad que el eje $r$ pertenece al plano $\pi$ perpendicular al vector $u = q - p$ y que contiene el punto $(p + q)/2$. Como aplicación, si $F$ es una rotación de $\mathbb{E}^{3}_{\mathbb{R}}$ tal que $F(1,1,1) = (1,1,0)$ y que $F(0,1,0) = (1,0,1)$, encontrad el eje, el ángulo de giro y las ecuaciones de $F$.}
\\

\hspace*{10mm} \textbf{(i) } Como $F$ es un movimiento, este mantiene distancias entre puntos, teniendo entonces que $d(p,q_1) = d(F(p),F(q_1)) = d(p,q_2)$. Por tanto $p,q_1,q_2$ forman un triángulo isósceles, con los lados adyacentes a $p$ del mismo tamaño. Consideremos la recta $r$ que pasa por $p$ y $(q_1+q_2)/2$. Sea $\alpha$ el ángulo del vértice $p$ en el triángulo $p,q_1,q_2$, entonces los ángulos respectivos a los dos otros vértices serán de $\frac{\pi - \alpha}{2}$. Consideremos ahora el triángulo $p,q_1,(q_1+q_2)/2$. El ángulo del vértice $p$ será lógicamente $\alpha /2$ por simetría, el ángulo del vértice $q_1$ se mantendrá igual que en el triángulo anterior, y por consecuente, el del vértice $(q_1+q_2)/2$ será $\pi - \frac{\alpha}{2} - \frac{\pi - \alpha}{2} = \pi/2$. Por tanto la recta $r$ tiene un ángulo de $\pi/2$ con $u$, por tanto es perpendicular a este vector, tal y como queríamos ver.

Aplicando lo que acabamos de demostrar, consideremos:
\begin{gather*}
    r: \frac{(1,1) + (-1,3)}{2} + [(-1,3) - (1,1)]^{\perp} = (0,2) + [(-2,2)]^{\perp} =(0,2) + [(1,1)]\\
    s: \frac{(2,0) + (0,4)}{2} + [(0,4) - (2,0)]^{\perp} = (1,2) + [(-2,4)]^{\perp} = (1,2) + [(2,1)]
\end{gather*}

Donde obviamente $<(-2,2),(1,1)> = -2 + 2 = 0$ y $<(-2,4),(2,1)> = -4 + 4 = 0$, por tanto son ortogonales. Una vez construidas las rectas $r,s$, siendo $p$ el centro del giro, $p\in r,s$. Por lo tanto, $p\in r \cap s$. Encontremos este punto a partir de las ecuaciones cartesianas de $r$ y $s$.

\begin{gather*}
    \begin{cases}
    r: x-y+2=0\\
    s: x-2y+3=0
    \end{cases}
    \Longrightarrow
    \begin{cases}
    x = y-2\\ y -2y + 1 = 0
    \end{cases}
    \Longrightarrow
    p = r \cap s =
    \begin{pmatrix}
    -1\\1
    \end{pmatrix}
\end{gather*}

Una vez encontrado el centro del giro, encontremos ahora el ángulo de giro de $F$. Sea $\alpha$ este ángulo, este será el ángulo entre las rectas $t: p + [\Vec{pq}]$ y $l: p + [\Vec{pF(q)}]$  $\forall q \in \mathbb{E}^{2}_{\mathbb{R}}$. Sea $q = (1,1)$ entonces:
\begin{gather*}
    cos(\alpha) = \frac{<\Vec{pq},\Vec{pF(q)}>}{\| \Vec{pq}\| \| \Vec{pF(q)} \|} = \frac{<(2,0),(0,2)>}{\|(2,0)\| \|(0,2)\|} = \frac{0}{4} = 0
\end{gather*}

Por tanto el giro tiene un ángulo de $\alpha = \pm \frac{\pi}{2}$. Veamos en qué sentido está girando para determinar el signo de $\alpha$.

\begin{gather*}
    v = (1,1) - (2,0) = (-1,1) \quad \quad F(v) = (-1,3)-(0,4) = (-1,-1)\\
    \\
    det(v,F(v)) = det
    \begin{pmatrix}
    -1 & -1\\ 1 & -1
    \end{pmatrix}
    = 1 +1 = 2 > 0
\end{gather*}

En consecuencia, el giro tiene un sentido antihorario $\Longrightarrow \alpha = \pi/2$. Así que en la base canónica la matriz de la parte lineal del movimiento tendrá la siguiente expresión:

\begin{gather*}
    \widetilde{F} = 
    \begin{pmatrix}
    cos\alpha & -sin\alpha\\
    sin\alpha & cos\alpha
    \end{pmatrix}
    =
    \begin{pmatrix}
    0 & -1\\
    1 & 0
    \end{pmatrix}
\end{gather*}

Y a partir de $q=(1,1)$ y $F(q) = (-1,3)$ tenemos

\begin{gather*}
    \begin{pmatrix}
    0 & -1\\
    1 & 0
    \end{pmatrix}
    \begin{pmatrix}
    1\\
    1 
    \end{pmatrix}
    +
    \begin{pmatrix}
    a\\
    b
    \end{pmatrix}
    =
    \begin{pmatrix}
    -1\\
    3
    \end{pmatrix}
    \Longrightarrow
    \begin{pmatrix}
    a\\b
    \end{pmatrix}
    =
    \begin{pmatrix}
    0\\2
    \end{pmatrix}
\end{gather*}

Donde obtenemos que $F = \widetilde{F} + \begin{pmatrix} 0\\2 \end{pmatrix}$. Por tanto, si $F \begin{pmatrix} x\\y \end{pmatrix} = \begin{pmatrix} x'\\y' \end{pmatrix}$, obtenemos como ecuaciones $F: \begin{cases}x' = -y\\ y' = x + 2 \end{cases}$  
\\
\\

\hspace*{10mm} \textbf{(ii)} Sea $v$ el vector director del eje $r$. Sea $\pi'$ el plano generado por $[v]^\perp$ tal que $p,q \in \pi'$. Claramente $\pi'$ es perpendicular a $r$ por como hemos construido el plano. Sea $c = \pi' \cap r$. Observamos que en este plano $\pi'$ nos encontramos en la misma situación que en \textbf{(i)}, por tanto, $\exists$ $s: c + [w] \subset \pi'$ tal que pasa por $c$ y por $\frac{p+q}{2}$ y es perpendicular a $u= q-p$ ($u \perp w$, con $u,w \in \pi'$). Si consideramos ahora el plano $\pi = c + [v,w]$, este contiene a $r$ y el punto $\frac{p+q}{2}$ y es perpendicular a $u$ ($u \perp w, w \perp v$ y $u \in [v]^{\perp}$ por como lo hemos construido), por tanto tenemos el plano $\pi$ que buscábamos.

Para encontrar los elementos característicos de $F$, consideremos los planos construidos a partir de los puntos y sus imágenes por $F$ que nos da el enunciado:
\begin{gather*}
    \pi_1= \frac{(1,1,1)+(1,1,0)}{2} + [(1,1,0) - (1,1,1)]^{\perp} = (1,1,1/2) + [(0,0,-1)]^{\perp}\\ = (1,1,1/2) + [(1,0,0),(0,1,0)]\\
    \pi_2 = \frac{(0,1,0)+(1,0,1)}{2} + [(1,0,1) - (0,1,0)]^{\perp} = (1/2,1/2,1/2) + [(1,-1,1)]^{\perp}\\ = (1/2,1/2,1/2) + [(1,1,0),(0,1,1)]
\end{gather*}

El eje de rotación $r \in \pi_1 \cap \pi_2$. Encontremos este punto a partir de las ecuaciones cartesianas de $\pi_1$ y $\pi_2$.

\begin{gather*}
    \begin{cases}
    \pi_1 : z - 1/2 = 0  \\
    \pi_2 : x -y + z -1/2 = 0
    \end{cases}
    \Longrightarrow
    \begin{cases}
    z = 1/2\\ x - y = 0
    \end{cases}
    \Longrightarrow
    r = \pi_1 \cap \pi_2 = (1,1,1/2) + [(1,1,0)]
\end{gather*}

Por tanto el eje de rotación es $r: (1,1,1/2) + [(1,1,0)]$. Ahora, para encontrar el ángulo de giro de la rotación consideremos:
\begin{gather*}
    \pi' = p + [v]^{\perp} = (1,1,1) +[(1,1,0)]^{\perp} = (1,1,1) + [(1,-1,0), (0,0,1)]\\ \\
    c = \pi' \cap r = 
    \begin{cases}
    r: \begin{cases} z= 1/2\\ x = y \end{cases}\\
    \pi': -x-y+2=0
    \end{cases}
    \Longrightarrow \quad
    \begin{cases}
    x = y\\ -2x = -2\\ z = 1/2
    \end{cases}
    \Longrightarrow \quad
    c = \begin{pmatrix} 1 \\ 1 \\ 1/2 \end{pmatrix}
\end{gather*}

Finalmente, si tenemos en cuenta las siguientes rectas $s,l \subset \pi'$, con $p=(1,1,1)$ y $q=(1,1,0)$:
\begin{gather*}
    s: c + [\Vec{cp}] = (1,1,1/2) + [(0,0,1/2)] \\
    l: c + [\Vec{cq}] = (1,1,1/2) + [(0,0,-1/2)]
\end{gather*}

El ángulo de rotación del movimiento será:

\begin{gather*}
    cos\alpha = \frac{<\Vec{cp},\Vec{cq}>}{\|\Vec{cp}\| \|\Vec{cq}\|} = \frac{<(0,0,1/2),(0,0,-1/2)>}{\|(0,0,1/2)\| \|(0,0,-1/2)\|} = \frac{-1/4}{1/4} = -1
\end{gather*}

Por tanto el ángulo de giro es $\pi$, con lo que esta rotación se podría considerar una simetría respecto a $r$. Esto ya lo podríamos haber deducido antes, ya que $s$ y $l$ son rectas coincidentes. Ahora, estudiando la parte lineal y teniendo en cuenta que $r$ es una recta de puntos fijos, tenemos:
\begin{gather*}
    \widetilde{F}(1,1,0) = \widetilde{F}(e_1 + e_2) = e_1 + e_2
\end{gather*}
Para el espacio ortogonal a $r$ tendremos que $\widetilde{F}(u) = -u$ por ser una simetría respecto a $r$. Así que,
\begin{gather*}
    \widetilde{F}(e_3) = -e_3\\
    \widetilde{F}(e_1 - e_2) = -e_1 + e_2
\end{gather*}
Por lo que obtenemos que

\begin{gather*}
    \widetilde{F}(e_1 + e_2 + e_1- e_2) = \widetilde{F}(e_1 + e_2) + \widetilde{F}(e_1 - e_2) = e_1 + e_2 - e_1 + e_2 = 2e_2\\ \Longrightarrow \widetilde{F}(2e_1) = 2e_2 \Longrightarrow \widetilde{F}(e_1) = e_2\\ \\\text{Luego, como }\widetilde{F}(e_1)=e_2 \text{ y } \widetilde{F}(e_1 + e_2)= e_1 + e_2 \Longrightarrow e_2 = e_1\\ \\
    \text{Finalmente tenemos } \widetilde{F}(e_3) = -e_3
\end{gather*}

Por lo que la matriz de $\widetilde{F}$ en la base canónica es:
\begin{gather*}
    M(\widetilde{F},e) = 
    \begin{pmatrix}
    0 & 1 & 0\\
    1 & 0 & 0\\
    0 & 0 & -1
    \end{pmatrix}
\end{gather*}

Ahora aplicando $F$ a (1,1,1) tenemos,

\begin{gather*}
    F(1,1,1) = 
    \begin{pmatrix}
    0 & 1 & 0\\
    1 & 0 & 0\\
    0 & 0 & -1
    \end{pmatrix}
    \begin{pmatrix} 1\\1\\1 \end{pmatrix}
    + \begin{pmatrix}a\\b\\c \end{pmatrix}
    = \begin{pmatrix} 1\\1\\0 \end{pmatrix}
    \Longrightarrow
    \begin{pmatrix}a\\b\\c \end{pmatrix}
    = \begin{pmatrix} 0\\0\\1 \end{pmatrix}
\end{gather*}

Por tanto, si $F(x,y,z) = (x', y', z')$ las ecuaciones de $F$ son: $x' = y, y' = x$ y $z' = -z + 1$.